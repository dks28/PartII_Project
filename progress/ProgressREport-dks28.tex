\documentclass[10pt, twoside, a4paper]{article}

\usepackage[margin=1in]{geometry}

\begin{document}

{\noindent \Large Part II Project Progress Report}
\vspace{0.6cm}

{\noindent \textbf{Name:}\ Daniel K.\ S\"a\"aw
 \quad     \textbf{CRSid:}\ dks28
 \quad     \textbf{College:}\ St Catharine's College
\vspace{1.0cm}

{\noindent \Large\bfseries On Spectral Methods for Clustering of Irregular Digraphs}
\vspace{0.6cm}

 \noindent \textbf{Supervisors:}\ Dr Luca Zanetti, Dr Thomas Sauerwald\\
	   \textbf{Overseers:}\ Prof. Anuj Dawar, Prof. Andrew Moore\\
	   \textbf{Director of Studies:}\ Dr Sergei Taraskin
}

\paragraph{Summary.}
Strictly speaking, I am almost on schedule; about one or two weeks behind my time line as presented in my original proposal. \\
In truth however, my progress on the project should be much larger. I had hoped to be ahead of my timetable significantly by this point after realising that the initial timeline was slightly slow, and that the work conducted should exceed the core criteria as proposed in Michaelmas term.  

The main reason that I have failed to meet my own expectations is rather complex and personal; I have been facing personal difficulties that, though unrelated to the project, have impeded its progress. The project itself has not proven more challenging than anticipated, as indicated in the above paragraph. 

A second factor in slowing down my progress on this project has been the way in which the department handled the industrial action at the end of Michaelmas with respect to my Michaelmas Unit of Assessment (Data Science: Principles and Practice). The workload for this Unit was expanded significantly and extended into Lent term; this has proven to enormously reduce the amount of attention I could pay to my project. 

\paragraph{Work Completed.}
I have fully met the first success cirterion of establishing a random-graph generating model that exhibits both crucial attributes: Configurable `directional communities' and highly heterogeneous (power-tail) in-degree distribution. This includes having developed a formal definition and developing a reasonably efficient algorithm to sample from the distribution. 

Furthermore, I have implemented highly efficient versions of the two main algorithms the project aims to subject to investigation, and verified that they function correctly by identifying an evaluation measure, implementing code for it and replicating previously published experiments. 

I have also planned the experimental set-up I shall be using to conduct the main study of the project, and completed first drafts for both the Introduction and Preparation chapters of my dissertation, as well as those parts of the implementation chapter concerning the progress made so far.

\paragraph{Further Work-Flow.}
I will aim to accelerate my work on the remainder of the project in order to catch up with the timeline, if necessary making use of the grace period at the end of Lent term that I included for the case that my project might be behind. This should certainly prove manable. By the time of the progress presentations, I will have begun (and perhaps concluded) conducting the first set of experiments that will then guide the remaider of the evalutaion study.

\end{document}
