\chapter{Original Proposal}


% This is all copied from the diploma one and honestly it's disgusting
\rightline{\censor{\large\emph{Daniel K. Sääw}}}
\medskip
\rightline{\censor{\large\emph{St Catharine's College}}}
\medskip
\rightline{\censor{\large\emph{dks28}}}

\vfil

\centerline{\large Computer Science Tripos Part II Project Proposal}
\vspace{0.4in}
\centerline{\Large \textbf{On Spectral Methods for Clustering of Irregular Digraphs}}
\vspace{0.3in}
\centerline{\large 24th October 2019}

\vfil

\textbf{Project Originators:} Dr Luca Zanetti, Dr Thomas Sauerwald, \censor{Daniel Sääw}

\ \\\ \\\ 
\textbf{Project Supervisors:} Dr Thomas Sauerwald and Dr Luca Zanetti

\ \\\ \\\ 
{\bf Director of Studies:} Dr Sergei Taraskin

\ \\\ \\\ 
{\bf Overseers:} Prof.\ Anuj Dawar and Prof.\ Andrew Moore

\vspace{0.5in}

\vfil
\eject

\section*{Introduction and Description of the Work}
	Clustering in general is the process of partitioning some data with the purpose of 
	grouping together data that is `similar' in some way. It has thus become a fundamental, 
	unsupervised, method in Machine Learning that allows an abstraction of some data set into 
	a (comparatively) small number of communities comprising some subset of the original data 
	the members of which are some notion of `similar' to one another. Due to results in 
	spectral graph theory (in particular, the fact that the algebra\"ic multiplicity of 0 
	in an undirected graph's Laplacian's characteristic polynomial is the number of connected 
	components in this graph, as well as the Davis-Kahan Theorem \cite{daviskahan}), 
	when clustering graphs with respect to, for example, conductance, we may apply spectral 
	algorithms. \par
	This means that it is fairly easy to cluster undirected graphs if by `communities' (i.e.\ 
	vertex subsets that are `similar') we mean well-connected internally compared to the 
	connections to the complement of the community. 
	However, it is less clear what a community might be in a directed setting, since the edge 
	direction makes notions of `connectivity' inherently more complex. For example, some 
	digraph $G$ might exhibit a `cyclic flow'
	structure in that the connections between individual subgraphs of $G$ might be 
	unidirectional, whilst each such subgraph has equal in- and out-degrees, meaning that the 
	undirected graph constructed from symmetrising $G$ might not exhibit communities in the 
	classical sense. In general, the directional information might crucially change the 
	meaning of the graph. Therefore, a number of techniques for clustering digraphs have 
	emerged based on notions of `directional communities'. These techniques are usually 
	supported with some model of directed random graph derived from Stochastic Block-Models 
	(SBMs), which have the perhaps unreasonable property that each node so generated graphs 
	will, in expectation, have equal in-degrees and out-degrees \cite{lucapaper, disimpaper}. \par
	This project, therefore, will involve attempting to develop a model of random digraphs that will 
	exhibit directional communities in a sense similar to previous directed SBMs, whilst 
	improving over previous such models by also exhibiting properties of real-world graphs not 
	included in previous models. For this purpose, a known model of random graph for an 
	undirected setting will 
	likely be augmented to feature directed edges to enable directional communities. Should 
	this fail, the project would fall back on investigating the performance of the proposed 
	algorithms on real-world graphs themselves.\\
	This model (or a set of real-world graphs with properties not modelled by previous digraph 
	models) will serve as a foundation for a comparison study of several spectral 
	techniques for clustering that will determine if the proposed techniques perform 
	satisfactory clustering on such `irregular' graphs. The random (or real-world) graphs used for this will have 
	the additional benefit that such graphs lend themselves to regularisation techniques 
	(again, previously developed for the setting of undirected graphs) -- that is modifications 
	to the underlying graphs that make certain properties (such as degree) more homogeneous 
	across the vertex set -- with the purpose of improving clusters detected by the algorithms,
	meaning that 
	the final step of the project will be conducting a second comparison study evaluating the 
	improvement of each algorithm under the employment of such regularisation techniques. 

\section*{Starting Point}
	The concepts of graph clustering and random (as well as real-world) graphs were briefly 
	introduced in the Paper 3 course \emph{Machine Learning and Real-world Data}. In that 
	course, however, the problem was approached programmatically where my project will employ 
	spectral methods. The linear algebra thus required for the project was touched on in the 
	course \emph{Mathematical Methods I} from the Natural Sciences Tripos, and linear algebra 
	in general is used in a number of courses of Part I of the Computer Science Tripos. 
	\par
	In the specific area of spectral clustering, there have been a few research papers 
	concerning the detection of `directional' communities, introducing models of random 
	digraphs with community structures, which should help extend undirected random graph models for this project. 
	The project will involve implementing and refining algorithms presented in \cite{lucapaper,
	disimpaper}. 
	Furthermore, there is some pre\"existing literature on regularising undirected graphs for 
	the purpose of clustering (e.g.\ \cite{rohereg, binyureg}), 
	from which I will draw a starting point for the corresponding 
	parts of the project.

\section*{Substance and Structure of the Project}
	The work carried out for this project will be the following:
	\begin{enumerate}
		\item A model of random graph will be developed
		(and failing that, existing collections of real-world graphs will be sampled to 
		replace the random model) that has the properties of 
		irregularity, in the sense that certain properties of the graphs will be strongly 
		heterogeneous across the graphs' vertices, and directional communities, in the sense that 
		certain subsets of vertices will exhibit a notion of similarity internally that 
		would be lost without directional information about the graph. This model will 
		draw inspiration from previous directed SBMs and might be influenced by properties 
		of a certain kind of real-world graph.
		\item Implementations of previously presented spectral algorithms for clustering 
		digraphs will be written, and comparatively evaluated in a statistically 
		significant setting to determine whether the proposed algorithms perform 
		satisfactory clustering with respect to the known, underlying communities inherent 
		in the model of random graph.
		\item Techniques for regularising graphs that have previously been developed for 
		undirected graphs will be applied (and if necessary, first ported to a directed 
		setting) to graphs drawn from the model developed at first, and the change in 
		performance of the implemented algorithms will be evaluated in a statistically 
		significant setting.
	\end{enumerate}

	\subsection*{Possible Extensions}
	The project might also...
	\begin{enumerate}[...]
		\item investigate a hypothesis formulated in some previous literature that 
		conjectures that using a graph representation that interpolates between the 
		directed and an abstracted, undirected, representation might improve clustering 
		results by retaining the directionality information whilst making the graph more 
		susceptible to classical spectral clustering techniques.
		\item apply the algorithms to real-world graphs that should intuitively be similar 
		to those generated by the random model, and evaluate the performance based on node 
		tags and determining the quality of the recovered clusters without a ground truth 
		for underlying clusters.
		\item develop formally a theoretical performance bound for the algorithms when 
		applied to the graphs sampled from the random model.
	\end{enumerate}

\section*{Success Criteria}
	The project should be deemed a success if the following criteria have been met:
	\begin{enumerate}
		\item Code implementing a (to be developed) model of random directed graph 
		with highly irregular degrees and directional communities has been written, 
		or a suitable set of real-world graphs with irregular features has been collected%
		\footnote{Data sets corresponding to, for example, citation networks in academia are
		available, and are prone to having the irregularity requirements (in the case of 
		citation networks, with respect to in-degree) as desired.}
		\item Spectral clustering algorithms proposed in previous research papers have 
		been implemented, both base versions and variants conducting initial graph 
		regularisation
		\item A statistically significant verdict has been reached about whether the 
		proposed algorithms are suitable for the clustering of highly irregular graphs 
		\item A statistically significant verdict has been reached about whether the 
		clusters detected by the proposed algorithms are improved by the application of 
		graph regularisation 
	\end{enumerate}

\section*{Timetable and Milestones}

I will divide my time into 2 week slots, to allow for sufficient granularity while simultaneously not committing myself to an excessively tight schedule.

\begin{enumerate}[\bfseries Slot 1 - \normalfont]
 \item 
 \emph{12th October -- 25th October} 
 \begin{itemize}
	\item finalise and submit project proposal
	\item research backgrounds of spectral clustering more thoroughly
 \end{itemize}
 
 \textbf{Deadlines}
 \begin{itemize}
  \item 
  Phase 1 Proposal Deadline -- 14th October, 3 PM.
  \item
  Draft Proposal deadline -- 18th October, 12 noon.
  \item
  Proposal Deadline -- 25th October, 12 noon.
 \end{itemize}
 
 \item 
 \emph{26th October -- 8th November}
 \begin{itemize}
	\item research models of random graphs with highly irregular degrees
	\item determine a general type of real-world graph that is of interest to decide details of
	random graph model
 \end{itemize}
 
 \item 
 \emph{9th November -- 22nd November}
 	\begin{itemize}
		\item formalise random-graph model
		\item begin planning components of experimental set-up
	\end{itemize}
 \item 
 \emph{23rd November -- 6th December} \\ \textit{Note: Due to my choice of Units of Assessment, this 
 period will likely be very stressful. I therefore foresee less time to work on the project during 
 this slot.}
 \begin{itemize}
	\item implement code to generate graphs drawn from the random model
 \end{itemize}
 
 \item 
 \emph{7th December -- 20th December}
 \begin{itemize}
	\item implement previously proposed spectral algorithms for digraph clustering
	\item implement code for graph normalisation
 \end{itemize}
 
 \item \label{measuredev}
 \emph{21st December -- 3rd January}
 \begin{itemize}
	\item develop a measure for extent of success of clustering procedure
	\item begin drafting introduction, preparation and implementation chapters of the 
	dissertation
 \end{itemize}
 
 \item 
 \emph{4th January -- 17th January}
 \begin{itemize}
	\item write code to generate graphs for test bench
	\item complete preparation for comparison study of algorithms, including generating a 
	set of graphs to compose the `test bench' to ensure internal consistency
 \end{itemize}
 
 \item 
 \emph{18th January -- 31st January}
 \begin{itemize}
	\item conduct experiments to evaluate clustering algorithms with regard to performance measure developed in slot \ref{measuredev}
	\item write and submit progress report
\end{itemize}
 
 \textbf{Deadlines}
 \begin{itemize}
  \item 
  Progress Report Deadline -- 31st January, 12 noon.
 \end{itemize}
 
 \item 
 \emph{1st February -- 14th February}
 \begin{itemize}
	\item begin evaluating results of experiments on unnormalised algorithms
	\item conduct experiments to evaluate performance change due to normalisation 
 	\item prepare progress report presentation
 \end{itemize}
 
 \textbf{Deadlines}
 \begin{itemize}
  \item 
  Progress Report Presentations -- 6th, 7th, 10th, 11th February, 2 PM.
 \end{itemize}
 
 \item 
 \emph{15th February -- 28th February}
 \begin{itemize}
	\item complete performance evaluation
	\item reach  verdict about statistical significance of results
 \end{itemize}
 
 \item 
 \emph{29st February -- 13th March}
 \begin{itemize}
	\item allow time to ensure that success criteria have been met
	\item consult with project supervisors on best continuing work (e.g.\ which of the 
	potential extensions would now be best approached)
 \end{itemize}
 
 \item 
 \emph{14th March -- 27th March}
 \begin{itemize}
	\item carry out further work according to supervisor consultation
	\item complete first chapters drafted in slot \ref{measuredev}
 \end{itemize}
 
 \item 
 \emph{28th March -- 10th April}
 \begin{itemize}
	\item write evaluation and conclusion chapters
	\item continue and complete any work that is yet to be completed
 \end{itemize}
 
 \item 
 \emph{11th April -- 24th April}
 \begin{itemize}
	\item complete draft dissertation
	\item revise details of work to improve overall quality of dissertation and ensure 
	consistency in dissertation
 \end{itemize}
 
 \item 
 \emph{25th April -- 8th May}
 \begin{itemize}
	\item allow for time to complete any remaining work in case this timeline has not been 
	accurate
	\item submit dissertation as early as possible to allow transition to revision
 \end{itemize}
 
 \textbf{Deadlines}
 \begin{itemize}
  \item 
  Dissertation Deadline -- 8th May, 12 noon.
  \item
  Source Code Deadline -- 8th May, 5 PM.
 \end{itemize}
\end{enumerate}

\section*{Resources Declaration}
	For the programming work to be done, I intend to use my own personal laptop. This has an
	Intel i7-7700HQ 4.0 GHz processor as well 16 gigabytes of random-access memory. To guard 
	against the case that my personal machine fails I shall back up all my work, including any 
	progress on the dissertation, using version control via GitHub, and regular commits to the 
	working repository. \\
	Should my computer thus fail, I should be able to transition fairly smoothly to the 
	machines provided by the managed cluster service (MCS). \\
	Experiments will be constructed, and any code required, using MatLab, a license for which
	the University provides me with; any graphs that need to be visualised will be visualised 
	using either MatLab or GePhi. \\
	Should the development of a model for random graphs fail, publically available real-world 
	data sets (yielding directed graphs) will be sampled to generate test benches for the 
	project instead of random graphs generated from an original random model.

