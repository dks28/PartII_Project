\section*{Pro Form{\ae}}
\ \\
\noindent
\begin{tabular}{@{}p{0.25\textwidth}p{0.75\textwidth}}
\bfseries Candidate Number:      & \ \\[5pt]
\bfseries Project Title:         & \title \\[5pt]
\bfseries Examination: 	       & Part II (Computer Science Tripos), Easter Term 2020\\[5pt]
\bfseries Word-count:            & 12 000\footnotemark \\[5pt]
\bfseries Final Line Count:      & 1 187\footnotemark\\[5pt]
\bfseries Project Originators:   & The candidate, Luca Zanetti and Thomas Sauerwald \\[5pt]
\bfseries Project Supervisors:   & Luca Zanetti and Thomas Sauerwald 
\end{tabular}
\addtocounter{footnote}{-1}
\footnotetext{Computed through \texttt{detex DissertationContent.tex | tr -cd '0-9A-Za-z {\textbackslash}n' | wc -w}} 
\addtocounter{footnote}{1}
\footnotetext{Computed using \texttt{cloc}}

\subsection*{Original Aims Of The Project}
Investigating the applicability of algorithms developed for the clustering of directed graphs into 
`directional communities' to graphs with extremely skewed degree distributions similar to many 
real-world networks and possibilities for improving this applicability using variants of 
`regularisation techniques' already well-known for the setting of undirected graphs.

Previously, these algorithms had (to the best of the candidate's knowledge) not been subjected to 
rigorous testing with the focus on such heterogeneous networks.

\subsection*{Work Completed}
In the scope of this project, the original aims of it have been met by 
\begin{enumerate}[(1)]
\item developing a formal definition of a random model for directed graphs modelling real-world degree distributions and including configurable directional communities as recovered by the investigated algorithms,
\item implementing efficient versions of the directed clustering algorithms,
\item developing a rigorous set of experiments to test the applicability of these algorithms to heterogeneous graphs and
\item extending this set of experiments for algorithms modified by regularisation techniques similar to ones used in undirected graphs.
\end{enumerate}

\subsection*{Special Difficulties}
Towards the end of Lent term 2020, governments around the world moved towards the closest to martial law any candidate for this examination will have experienced in their lifetime in order to counteract the spread of SARS-CoV-2, and all candidates found themselves members of a department that in response acted as if nothing extraordinary had occurred.
