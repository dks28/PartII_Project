\chapter{Conclusions}

In this project, I have aimed to present a thorough and discerning evaluation of directed-clustering
algorithms that have been developed in recent years, by producing an environment that does not 
favour either algorithm or gives any reason to expect unreasonably good performance, as well as 
provide a reproducible set of experiments, each of which finds justification in the realm of 
real-world machine-learning tasks.

This has been necessary due to the relatively unexplored nature of such directed-graph clustering 
algorithms, combined with the ubiquity of directed graphs in the real world and their resulting 
relevance to machine-learning tasks. Thus, to accommodate the nature of this project as an 
exploratory survey of the applicability of the existing algorithms I have had to justify each 
decision throughout the project with the relevance to real-world application in mind, as I have 
successfully done as described in the previous chapter.

\section*{Future Work}
I have talked previously about the ubiquity of directed networks with tail-heavy degree 
distributions in the real world. Alas, this project has not had the time to investigate any 
real-world networks. This would be the most obvious next step in refining directed-clustering 
algorithms. 

In addition, since the previously used regularisation techniques proved not to be as applicable in 
the directed setting, it should be an interesting research question  to try and determine either 
theoretical or empirically justified solutions to improving performance for this task, which could 
provide the basis for a project in Part III of the Computer Science Tripos, or the MPhil in 
Advanced Computer Science similar to this project with a slightly different focus.


