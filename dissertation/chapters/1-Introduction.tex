%\usepackage[backend=biber]{biblatex}

\chapter{Introduction}

Graph clustering is a common task in unsupervised machine learning.

Given a graph $G = (V, E)$, it is finding a partition of its vertex set $V$ such that vertices are 
grouped into communities (or `clusters') that maximise some notion of 
similarity \cite{jain1988algorithms}. It is important in unupervised machine learning since it can 
reveal underlying structures in networks that could serve as abstractions of the graph or reveal 
previously unknown families of elements of the network.

In simple, undirected graphs, this notion of similarity is usually taken to be some 
measure of well-connectedness, for example the conductance of the resulting partition \cite{conductance}. In that 
setting, the problem is well-understood due to the rather simple formulation of what makes nodes 
similar. 

However, the same problem is inherently much harder when considering directed graphs (digraphs), 
as the notion of node similarity becomes less clear. One approach to handling digraphs is ignoring 
edge directionality, effectively reducing the problem to its easier relative. This is not 
satisfactory in most cases as the significance of an edge within a network can be drastically 
impacted by its direction. For example, in a digraph modelling trade between various nations, it 
becomes crucial to the information conveyed by the graph, and reasonable notions of what makes a 
nation similar to another economically, which direction trade occurs in. In full generality, by 
removing edge directions, information is lost leading to generally sypeaking weaker clustering 
results. 

Therefore, an alternative way of dealing with digraphs needs to be developed, together with an 
alternative notion of vertex similarity in a directed setting. In general, therefore, instead of
well-connectedness, communities in digraphs are determined by either similar sending or receiving 
patterns of their nodes. When applied to the example of international trade, this would mean 
countries with a tendency to export to some particular nation whilst importing from another might 
be clustered together. \\\ 

These notions of similarity have to be captured by the representation that is used for the graph 
in the clustering algorithm. In particular, algorithms using the spectrum of some construction on 
the graph's adjacency matrix (known as \emph{spectral} algorithms) derive the detailed 
representation of the input graph through the notion of similarity they intend to focus on in 
particular. Two particular such algorithms have recently been developed, with slightly different 
intents. These algorithms' developers have also introduced their own random-graph models with 
configurable directional communities that were used to develop and test the clustering algorithms \cite{lucapaper, disimpaper}.

These graph models exhibit a flaw, however. In the real world, many networks exhibit heterogeneous 
distributions of the (in-/out-)degrees of their nodes, while the models developed in previous 
research have rather homogeneous degree distributions. Using spectral-clustering algorithms on 
graphs with irregular degree distributions exhibits a significant performance hit in the undirected 
setting, so regularistation techniques have been applied to such irregular unirected graphs to
remedy this; as such similar techniques may be applicable to digraphs for the same purpose as well \cite{rohereg, binyureg}.
